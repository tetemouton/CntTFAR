\begin{table}[h]
\begin{tabular}{ | p{5cm} | p{10cm} |}
\hline
Depletion & Depletion describes the level of reduction in the fish stock since fishing
first began, typically by comparing current spawning biomass to that
which would occur if there was no fishing (SBcurrent/SBF =0). \\ \hline
Fishing mortality rate & The proportion of the stock removed by fishing in a unit of time.
Growth overfished Fish are harvested at an average size that is smaller than the size that
would produce the maximum yield per recruit. \\ \hline
Maximum Sustainable Yield (MSY) & The maximum amount of catch that can be taken from the stock per
year, on average, in the long-term. \\ \hline
Overfished & Occurs when there are no longer enough adults in the population to
produce enough young to replace those fish removed from the population
by fishing. In the WCPFC, an overfished fishery is defined as one
where the current spawning biomass (SBcurrent) is less than 20 of the
spawning biomass in the absence of fishing (SBF =0). \\ \hline
Overfishing & In the WCPFC, overfishing is defined as occurring when the current
fishing mortality rate exceeds the fishing mortality rate that would
provide the maximum sustainable yield. Sustained overfishing leads to
an overfished state. \\ \hline
Recruitment overfished & Occurs when the adult population is depleted to a level where it no longer has the reproductive capacity to replenish itself. \\
\hline
\end{tabular}
\caption{Definitions of key terms used in describing the impact of fishing upon and the status of fish stocks} 
\end{table}